\documentclass{../tex/report}
\usepackage{setspace} % Setting line spacing
\usepackage{ulem} % Underline
\usepackage{caption} % Captioning figures
\usepackage{subcaption} % Subfigures
\usepackage{geometry} % Page layout
\usepackage{multicol} % Columned pages
\usepackage{array,etoolbox}
\usepackage{fancyhdr}
\usepackage{enumitem}
\usepackage[toc,page]{appendix}

% Page layout (margins, size, line spacing)
\geometry{letterpaper, left=1in, right=1in, bottom=1in, top=1in}
\setstretch{1.5}

% Headers
\pagestyle{fancy}
\lhead{I${}^2$CIP - Requirements}
\rhead{PeaPod Technologies Inc.}

% Metric counter, referencing commands
\newcounter{metricnumber}
\setcounter{metricnumber}{1}
\newcommand{\metricrow}{M\arabic{metricnumber}}
\newcommand{\mlabel}[1]{\addtocounter{metricnumber}{-1}\refstepcounter{metricnumber}\label{#1}\addtocounter{metricnumber}{1}}
\newcommand{\mref}[1]{\hyperref[#1]{M\ref{#1}}}

\begin{document}

\begin{titlepage}
    \begin{center}
        \vspace*{1.2cm}

        \textbf{\large{I${}^2$CIP: Inter-Integrated Circuit Intra-networking Protocols}}

        \vspace{0.5cm}

        Requirements for: a Hardware Design Specification for Bus-Switched Hot-Swap Modules of I${}^2$C Targets, and; a Software Library of Intra-Network Communications Protocols for Rapid Implementation of Plug-and-Play Embedded Systems\\

        \vfill
        \scriptsize{
            \textbf{Jayden Lefebvre - Founder \& CEO, Lead Engineer}\\
            Northumberland County, ON, Canada\\
        }
        \vspace{.75cm}
        Primary Contact Email: \uline{contact@peapodtech.com}
        \vspace{1.25cm}

        Revision 1.0\\
        PeaPod Technologies Inc.\\
        July 26th, 2024

    \end{center}
\end{titlepage}

\thispagestyle{plain}

\tableofcontents
\newpage

\section{Introduction}\label{sec:intro}

\subsection{Purpose}\label{sec:purpose}

The purpose of this document is to two-fold:

Outline the categorical requirements (Section \ref{sec:requirements}) of an intra-network design specification for hot-swap bus-switched modules of plug-and-play I${}^2$C devices, and a software library for communications protocols

Detail the scoped requirements (Section \ref{sec:scope}) for the design and protocols as-proposed by PeaPod Technologies Inc., namely \textbf{I${}^2$CIP}.

% conditions:

% - Dynamic Routing: ***.*** This enables physical “plug-and-play” functionality.
% - Modularity: ***.*** This enables lifecycle management for physical collections of devices, and informs dynamic routing.

\subsection{Design Paradigm}\label{sec:structure}

% \subsection{Problem Framing and Project-Design Paradigm}\label{sec:paradigm_framing}

All Solutions provided by PeaPod Technologies are fully documented and backed by our rigorous top-down engineering-design paradigm, applied at all steps, enabling us to frame the Problem in such a way that a Solution is a provable and self-evident combination of our Services:

\begin{itemize}
    \item \textbf{\ref{sec:opportunity} - Problem Statement}: A \uline{scoped overview} of the opportunity or \textbf{Problem}. Anything not covered by this \textbf{Statement} \uline{is not considered} in the execution of the project.
    \item \textbf{\ref{sec:requirements} - Solution Requirements}: Categorical requirements for \textit{any} solution to the problem, interpolated from the scope boundaries in the \textbf{Problem Statement}. If any of these are not met, the problem is not solved, providing a distinct \textit{pass/fail threshold}, or "razor", for possible Solutions.
    \item \textbf{\ref{sec:stakeholders} - Stakeholders and Values}: \textit{Perspectives} in consideration (i.e. ITCGI, the client, client-of-client), along with a summarized \textit{value proposition} for each person or group, derived from the \textbf{Problem Statement} and \textbf{Requirements}. These are the people who are and will be \textit{directly affected} by the \textbf{Problem}, and as such their \textbf{Values} will influence the \uline{selection of factors} about which we shall select the ideal \textbf{Solution}.
    \item \textbf{\ref{sec:goals} - Problem-Solving Goals}: \textit{Conceptual} design goals (e.g. Safety, Efficiency) derived from \textbf{Requirements} and \textbf{Stakeholder Values}.
    \item \textbf{\ref{sec:objectives} - Solution Objectives}: \textit{Tactical} implementation targets derived from the \textbf{Requirements} and \textbf{Goals}.
    \item \textbf{\ref{sec:metrics} - Metrics}: Granular, \textit{quantitative} measures of design success/fit/utility/etc. derived from the \textbf{Objectives}, which are either \uline{Constrained} or \uline{Graded} according to the \textbf{Requirements}, \textbf{Goals}, and \textbf{Objectives}.
    \item \textbf{\ref{sec:constraints} - Constraints}: \textit{Mandatory} thresholds (i.e. true/false, pass/fail) and extrema (minima, maxima) for evaluating and \uline{disqualifying} proposed solutions along \uline{Constrained} \textbf{Metrics}.
    \item \textbf{\ref{sec:criteria} - Criteria}: \textit{Points-based} system for evaluating and \uline{ranking} proposed solutions along \uline{Graded} \textbf{Metrics}.
\end{itemize} 

\clearpage


\subsection{Scope and Justification}\label{sec:scope}

\subsubsection{I${}^2$C Specification}

From the I${}^2$C Specification Version 7 (2021, \textit{NXP Semiconductors}): 8-bit-oriented one-ended (controller-driven) bidirectional (read \& write) serial communication over a 2-wire bus (data "SDA" \& clock "SCL") for integrated circuit "devices" ("targets"), including (but not limited to):

\begin{itemize}
    \item \textbf{Remote Multi-Channel Ports}: GPIO banks, internal-clock PWM drivers, A/D and D/A converters, etc.
    \item \textbf{System Devices}: real-time clocks, LCD screens, etc.
    \item \textbf{Data Storage Devices}: EEPROM, SRAM, FRAM, etc.
    \item \textbf{Digital Sensors}: temperature, light, speed, pressure, etc.
\end{itemize}

The I${}^2$C Specification can be imagined as an incomplete analogue to the Internet's OSI Model, with the following layers defined:

\begin{enumerate}[label=SC\arabic*., ref=SC\arabic*]
    \item\label{sc:1} \textbf{Physical Layer} - Compare to RJ45
    \begin{enumerate}[ref=SC1\alph*]
        \item\label{sc:1a} \textbf{VDD \& GND} - +5 VDC
        \item\label{sc:1b} \textbf{SDA \& SCL} - Pull-Up Bias Resistors (10 k$\Omega$)
    \end{enumerate}
    \item\label{sc:2} \textbf{Data Link Layer} - Compare to Ethernet
    \begin{enumerate}[ref=SC2\alph*]
        \item\label{sc:2a} \textbf{Controller} - Bus Speed Control, Start \& Stop Conditions, Multi-Controller Arbitration
        \item\label{sc:2b} \textbf{Targets} - 7-bit Device Addressing, Acknowledgement ("ACK")
        \item\label{sc:2c} \textbf{Packet Structure} Read \& Write Flags, 8-/16-bit Register Addressing, Byte-Stream Data
    \end{enumerate}
\end{enumerate}

The \textbf{Network} (data routing), \textbf{Transport} (data delivery), and \textbf{Session} (transmission context) layers of the OSI model analogy are not defined by the I${}^2$C Specification. The following proposed extensions to the I${}^2$C Specification, the focus of the \textbf{I${}^2$CIP} design, are intended to fill this gap, enabling \textbf{Presentation} and \textbf{Application} layer functionality to be rapidly implemented by developers for embedded systems.

% TODO: WHy modules?
% TODO: Use for control systems

\subsubsection{Switched-Bus Intra-Networking}

Suppose an I${}^2$C target device $D$ with $N$ possible unique addresses. An I${}^2$C bus controller can communicate with $N$ uniquely-addressed instantiations of $D$ on one bus without conflict or modification to connections.

Suppose an I${}^2$C target device with $M$ possible unique addresses that acts as a multiplexer and repeater ("switch") for $B$ bitwise-enabled output busses ("subnets"). An I${}^2$C bus controller can communicate with $M * B * N$ independently-addressable instantiations of $D$ across $M * B$ subnets by setting ONE active output bus on each of the $M$ switches (and disabling ALL on the remaining $M - 1$).

Example: For $M = 8$ switches with $B = 8$ subnets each, the total number of uniquely-addressable instantiations of $D$ on an I${}^2$CIP intra-network is $8 * 8 * N = 64N$, effectively enabling a 64-fold increase in independently-addressable targets of all types by a single controller.

% Module = switch + routing table

\begin{enumerate}[label=SC\arabic*., ref=SC\arabic*, resume]
    \item\label{sc:3} \textbf{Network Layer} - Fully-Qualified Addressing ("FQA") - Compare to IP Addressing
    \item\label{sc:4} \textbf{Transport Layer} - Switch \& Target Ping and Target Control with Quality-of-Service 2 ("only-once" delivery) via ACK - Compare to TCP
    \item\label{sc:5} \textbf{Session Layer} - Target Discovery \& Module Configuration via Dedicated EEPROM Target
\end{enumerate}

% Electrical Isolation (\textit{ISO1540})



% Unimplemented, but considered for future development:
% \begin{enumerate}[label=SC\arabic*., ref=SC\arabic*, resume]
%     \item\label{sc:6} \textbf{Presentation Layer} - Data Formatting, Data Parsing
%     \item\label{sc:7} \textbf{Application Layer} - Data Conversion, Data Processing
% \end{enumerate}

% Eight modules, eight busses per module
% "DISABLED" Bus - prevents cross-talk

\subsection{Definitions}\label{sec:definitions}

A number of useful definitions have emerged from the above scoping:
\begin{enumerate}
    \item \textbf{Switch} - An I${}^2$C target device that acts as a repeater for bitwise multiplexed output busses.
    \item \textbf{Subnet} - A specific output bus of a specific switch.
    \item \textbf{Intra-Network} - A general term referring to ALL routable targets (not including switches) on ALL subnets across ALL of a controller's I${}^2$C busses (in the case of multiple controller busses).
    \item \textbf{Fully-Qualified Address (FQA)} - A unique intra-network routing location identifier, encoding: a specific I${}^2$C bus (in the case of multiple controller busses), and; a specific Subnet, and; a target's address.
    \item \textbf{Module} - A switch, and; a physical collection of targets located on the switch's subnets, and; a data storage target with a predetermined address with all routing information for all targets on this switch's subnets.
\end{enumerate}

\clearpage


\section{Framing}\label{sec:framing}

\subsection{Problem Statement}\label{sec:opportunity}



\subsection{Solution Requirements}\label{sec:requirements}




% Justify:
% I2CIP MUST provide the context for the implementation for Module lifecycle management
% I2CIP MUST provide the context for the implementation for Module configuration via EEPROM
% I2CIP DOES NOT provide the implementation of Module data parsing or formatting or conversion (i.e. data is passed as-is as bytes)


The following are the overall challenge requirements compiled from A, B, and an excerpt from C:
\begin{enumerate}[label=R\arabic*., ref=R\arabic*]
    \item\label{r:1} \textbf{Must} lorem ipsum dolor sit amet, consectetur adipiscing elit:
    \begin{enumerate}[ref=R1\alph*]
        \item\label{r:1a} \textbf{Should} lorem ipsum dolor sit amet, consectetur adipiscing elit;
    \end{enumerate}
\end{enumerate}

% Change line spacing for the more list-heavy sections
\setstretch{1}
\subsection{Stakeholders and Values}\label{sec:stakeholders}

\begin{enumerate}[label=S\arabic*., ref=S\arabic*]
    \item\label{s:1} A - Values, etc.
    \item\label{s:2} B - DfX, etc.
\end{enumerate}

\clearpage


\subsection{Problem-Solving Goals}\label{sec:goals}

% High-Level
\begin{multicols}{2}[]
    \begin{enumerate}[label=HL\arabic*., ref=HL\arabic*]
        \item\label{hl:output} Goal ABC \hfill (\ref{s:1},~\ref{r:1},~\ref{r:1a})
    \end{enumerate}
\end{multicols}

\subsection{Solution Objectives}\label{sec:objectives}

% I2CIP DOES provide the the context for the implementation of Device parsing (but not formatting or conversion), as exemplified in the classes `EEPROM' (i.e. implementations of virtual functions inherited from Device and IOInterface<Get: char* readRegistersFromZero(uint16_t lenUntilNull), Set: const char* writeRegistersFromZero(uint16_t lenUntilNull)>)

% Low-Level
\begin{multicols}{2}[]
    \begin{enumerate}[label=LL\arabic*., ref=LL\arabic*]
        \item\label{ll:output_variety} Objective ABC \hfill (\ref{hl:output})
    \end{enumerate}
\end{multicols}

\clearpage


\subsection{Metrics}\label{sec:metrics}

\begin{tabular}{| @{\makebox[2.4em][c]{\metricrow}} | p{8.7cm} | p{5.9cm} |} 
    \hline
    \multicolumn{1}{| @{\makebox[2.4em][c]{\textbf{\#}}} | l |}{\textbf{Metric}} & \textbf{Units}\\ 
    \hline
    Metric ABC \mlabel{m:constraint} \hfill (\ref{ll:output_variety}) & Yes/No \\
    Metric XYZ \mlabel{m:criteria} \hfill (\ref{ll:output_variety}) & 0 - 100\% \\
    \hline
\end{tabular}

\clearpage


\subsection{Constraints}\label{sec:constraints}

\begin{tabular}{|l|p{14.35cm}|}
    \hline
    \textbf{Metric} & \textbf{Constraint \hfill Justification} \\
    \hline
    \mref{m:constraint} & Yes \hfill (\ref{s:1})\\
    \hline
\end{tabular}

\subsection{Criteria}\label{sec:criteria}

\begin{tabular}{|l|p{14.35cm}|}
    \hline
    \textbf{Metric} & \textbf{Criteria \hfill Justification} \\
    \hline
    \mref{m:criteria} & Should Maximize \hfill (\ref{r:1a}) \\
    \hline
\end{tabular}

% Refer to Appendix \ref{sec:assessment} for prototype verification Assessment Criteria (categories, weights, etc.).

\newpage

\section{Reference Designs}

% -------- TEMPLATE --------
% Introduction - Project goal, scope, differences from this project
% Graphics - Design drawings/photos, etc.
% Analysis - Rank the design across each of our metrics
    % TODO: Metrics might be too much, maybe just qualitative analysis based on LLOs?

\subsection{Reference Design XYZ}

Lorem ipsum dolor sit amet, consectetur adipiscing elit. Sed auctor, nunc nec ultricies ultricies, nunc nunc ultricies nunc, nec ultricies nunc nunc nec.

% \newpage

% % References
% \bibliographystyle{IEEEtran}
% \bibliography{references}

\end{document}