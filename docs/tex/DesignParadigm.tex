% \subsection{Problem Framing and Project-Design Paradigm}\label{sec:paradigm_framing}

All Solutions provided by PeaPod Technologies are fully documented and backed by our rigorous top-down engineering-design paradigm, applied at all steps, enabling us to frame the Problem in such a way that a Solution is a provable and self-evident combination of our Services:

\begin{itemize}
    \item \textbf{\ref{sec:opportunity} - Problem Statement}: A \uline{scoped overview} of the opportunity or \textbf{Problem}. Anything not covered by this \textbf{Statement} \uline{is not considered} in the execution of the project.
    \item \textbf{\ref{sec:requirements} - Solution Requirements}: Categorical requirements for \textit{any} solution to the problem, interpolated from the scope boundaries in the \textbf{Problem Statement}. If any of these are not met, the problem is not solved, providing a distinct \textit{pass/fail threshold}, or "razor", for possible Solutions.
    \item \textbf{\ref{sec:stakeholders} - Stakeholders and Values}: \textit{Perspectives} in consideration (i.e. ITCGI, the client, client-of-client), along with a summarized \textit{value proposition} for each person or group, derived from the \textbf{Problem Statement} and \textbf{Requirements}. These are the people who are and will be \textit{directly affected} by the \textbf{Problem}, and as such their \textbf{Values} will influence the \uline{selection of factors} about which we shall select the ideal \textbf{Solution}.
    \item \textbf{\ref{sec:goals} - Problem-Solving Goals}: \textit{Conceptual} design goals (e.g. Safety, Efficiency) derived from \textbf{Requirements} and \textbf{Stakeholder Values}.
    \item \textbf{\ref{sec:objectives} - Solution Objectives}: \textit{Tactical} implementation targets derived from the \textbf{Requirements} and \textbf{Goals}.
    \item \textbf{\ref{sec:metrics} - Metrics}: Granular, \textit{quantitative} measures of design success/fit/utility/etc. derived from the \textbf{Objectives}, which are either \uline{Constrained} or \uline{Graded} according to the \textbf{Requirements}, \textbf{Goals}, and \textbf{Objectives}.
    \item \textbf{\ref{sec:constraints} - Constraints}: \textit{Mandatory} thresholds (i.e. true/false, pass/fail) and extrema (minima, maxima) for evaluating and \uline{disqualifying} proposed solutions along \uline{Constrained} \textbf{Metrics}.
    \item \textbf{\ref{sec:criteria} - Criteria}: \textit{Points-based} system for evaluating and \uline{ranking} proposed solutions along \uline{Graded} \textbf{Metrics}.
\end{itemize} 